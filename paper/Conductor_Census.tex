\documentclass[11pt,a4paper]{article}
\usepackage[margin=2.5cm]{geometry}
\usepackage{amsmath,amssymb,amsthm}
\usepackage{booktabs}
\usepackage{graphicx}
\usepackage{hyperref}
\usepackage{float}
\usepackage{url}

\newcommand{\rad}{\operatorname{rad}}
\newcommand{\Jac}{\operatorname{Jac}}
\newcommand{\Cond}{\operatorname{Cond}}
\newcommand{\GSp}{\mathrm{GSp}}

\newtheorem{theorem}{Theorem}[section]
\newtheorem{proposition}[theorem]{Proposition}
\newtheorem{corollary}[theorem]{Corollary}
\newtheorem{conjecture}[theorem]{Conjecture}
\theoremstyle{definition}
\newtheorem{definition}[theorem]{Definition}
\theoremstyle{remark}
\newtheorem{remark}[theorem]{Remark}

\title{A Conductor Census of 425,082 Goldbach--Frey Curves:\\
  Asymptotic Decomposition and Bandwidth Stability}
\author{Ruqing Chen\\[4pt]
  \small GUT Geoservice Inc., Montreal\\
  \small\texttt{ruqing@hotmail.com}}
\date{February 2026}

\begin{document}
\maketitle

\begin{abstract}
Using the exact conductor formula
$\Cond_{\mathrm{odd}}(\Jac(C)) =
[\rad_{\mathrm{odd}}(p \cdot q \cdot N \cdot (p-q))]^2$
established in the preceding papers of this series, we compute
the conductor ratio~$\rho$ for every Goldbach pair $(p, q)$
with $p + q = 2N \le 10{,}000$: a census of 425,082~curves.
We decompose the mean conductor ratio into three components:
a static term~$\rho_{\mathrm{static}}$ depending only on~$N$
(6.5\% of~$\langle\rho\rangle$), a boundary
term~$\rho_{\mathrm{boundary}}$ from $p \cdot q$
(66.0\%), and a gap term~$\delta$ from $|p - q|$
(27.5\%).  The key empirical finding is \emph{bandwidth stability}:
the standard deviation $\sigma(\rho) \approx 0.33$ is approximately
constant over the entire range $2N \in [100,\; 10{,}000]$, while
$\langle\rho\rangle$ grows logarithmically.  This implies
the coefficient of variation $\sigma/\langle\rho\rangle
\to 0$ as $N \to \infty$, providing a quantitative explanation
for the high $R^2 > 0.997$ of the Band Shifting Law and
connecting the conductor geometry to the Hardy--Littlewood
prime pair density.
\end{abstract}


% ═══════════════════════════════════════════════════════════════════════════
\section{Introduction}

Papers~\cite{Chen2026TCV} and~\cite{Chen2026UTS} established the
exact odd conductor formula for the Goldbach--Frey curve
$C\colon y^2 = x(x^2 - p^2)(x^2 - q^2)$, with $p \ne q$
distinct odd primes and $p + q = 2N$:
\begin{equation}\label{eq:cond}
  \Cond_{\mathrm{odd}}\bigl(\Jac(C)\bigr)
  \;=\;
  \bigl[\rad_{\mathrm{odd}}(p \cdot q \cdot N
    \cdot (p - q))\bigr]^2.
\end{equation}
The conductor ratio
\begin{equation}\label{eq:rho}
  \rho(p, q)
  \;=\;
  \frac{\log \Cond_{\mathrm{odd}}\bigl(\Jac(C)\bigr)}{\log(2N)}
  \;=\;
  \frac{2\log\rad_{\mathrm{odd}}
    (p \cdot q \cdot N \cdot (p - q))}{\log(2N)}
\end{equation}
drives the Band Shifting Law (BSL) with $R^2 > 0.997$.
But the preceding work studied only 10~test cases.  The
present paper asks: what can we learn from \emph{all} Goldbach
pairs up to a given bound?

We compute $\rho(p, q)$ for every Goldbach pair with
$p + q \le 10{,}000$---a total of 425,082~pairs spanning
4,997~distinct even integers---and study the statistical
distribution of~$\rho$ as a function of~$N$.


% ═══════════════════════════════════════════════════════════════════════════
\section{Three-Component Decomposition}\label{sec:decomp}

The conductor ratio admits a natural decomposition.
Since $p$ and~$q$ are prime, $\rad_{\mathrm{odd}}(p) = p$
and $\rad_{\mathrm{odd}}(q) = q$.  Write
\begin{equation}\label{eq:decomp}
  \rho \;=\; \underbrace{\frac{2\log\rad_{\mathrm{odd}}(N)}
    {\log(2N)}}_{\rho_{\mathrm{static}}}
  \;+\;
  \underbrace{\frac{2\log(p \cdot q)}{\log(2N)}}
    _{\rho_{\mathrm{boundary}}}
  \;+\;
  \underbrace{\frac{2\log\rad_{\mathrm{odd}}^{\mathrm{new}}
    (|p - q|)}{\log(2N)}}_{\delta}\,,
\end{equation}
where $\rad_{\mathrm{odd}}^{\mathrm{new}}(|p - q|)$ denotes
the product of odd primes in $|p - q|$ not already dividing
$p \cdot q \cdot N$.

The three terms have distinct roles:

\begin{itemize}
\item
  $\rho_{\mathrm{static}}$: depends \emph{only} on~$N$.
  This is the static conduit $\xi = 2\log\rad_{\mathrm{odd}}(N)
  / \log(2N)$ that sets the \emph{band position} in the BSL.
\item
  $\rho_{\mathrm{boundary}}$: depends on the specific pair
  $(p, q)$ but not on their difference.
  For ``balanced'' pairs ($p \approx q \approx N$),
  $\rho_{\mathrm{boundary}} \approx 4\log N / \log(2N) \to 4$.
  For extreme pairs ($p = 3$, $q \approx 2N$),
  it is smaller.
\item
  $\delta$: the gap correction, a chaotic term depending on the
  prime factorisation of $|p - q|$.
  This contributes only scatter (``bandwidth'') to the BSL.
\end{itemize}

Table~\ref{tab:decomp} and Figure~\ref{fig:decomp}(b) show the
relative contributions at $2N = 10{,}000$.

\begin{table}[H]
\centering
\begin{tabular}{lrl}
\toprule
Component & Contribution & Role in BSL \\
\midrule
$\rho_{\mathrm{static}}$
  & 6.5\% & Band position \\
$\langle\rho_{\mathrm{boundary}}\rangle$
  & 66.0\% & Pair location within band \\
$\langle\delta\rangle$
  & 27.5\% & Intra-band noise \\
\bottomrule
\end{tabular}
\caption{Decomposition of $\langle\rho_{\mathrm{true}}\rangle
  = 5.38$ at $2N = 10{,}000$ (127~Goldbach pairs).}
\label{tab:decomp}
\end{table}


% ═══════════════════════════════════════════════════════════════════════════
\section{Census Results}\label{sec:results}

\subsection{Mean conductor ratio}

Figure~\ref{fig:decomp}(a) shows
$\langle\rho_{\mathrm{true}}\rangle$ versus $\ln(2N)$ for all
4,997~even integers.  The mean grows slowly and roughly
logarithmically; a linear fit in $\ln(2N)$ gives
\begin{equation}\label{eq:fit-rho}
  \langle\rho_{\mathrm{true}}\rangle
  \;\approx\; 0.217\,\ln(2N) + 4.55
\end{equation}
but with $R^2 = 0.33$, reflecting the strong number-theoretic
fluctuations in $\rad_{\mathrm{odd}}(N)$.

\begin{figure}[H]
  \centering
  \includegraphics[width=\textwidth]{../figures/fig_mean_decomp.pdf}
  \caption{%
    (a)~Mean conductor ratio $\langle\rho\rangle$ vs.\ $\ln(2N)$
    for 4,997~even numbers (blue dots), with moving average (red).
    (b)~Stacked decomposition: static (green),
    boundary (blue), gap (red).}
  \label{fig:decomp}
\end{figure}


\subsection{Bandwidth stability}

The most striking finding is that the bandwidth---the standard
deviation $\sigma(\rho)$ of the conductor ratio across Goldbach
pairs at fixed~$N$---is approximately constant:
\begin{equation}\label{eq:bandwidth}
  \sigma(\rho) \;\approx\; 0.33 \pm 0.05
  \qquad\text{for } 2N \in [100,\; 10{,}000].
\end{equation}
This is visible in Figure~\ref{fig:bw}(a).
Since $\langle\rho\rangle$ grows (slowly) while $\sigma$ stays
fixed, the coefficient of variation
$\mathrm{CV} = \sigma / \langle\rho\rangle$ decreases
(Figure~\ref{fig:bw}(b) of the Hardy--Littlewood figure).

\begin{figure}[H]
  \centering
  \includegraphics[width=\textwidth]{../figures/fig_bandwidth_norm.pdf}
  \caption{%
    (a)~Bandwidth $\sigma(\rho)$ vs.\ $2N$: approximately
    constant at $\approx 0.33$.
    (b)~Normalised ratio $\langle\rho\rangle / \ln(2N)$,
    decreasing toward a value near $\tfrac{1}{2}$.}
  \label{fig:bw}
\end{figure}


\subsection{Hardy--Littlewood comparison}

The Goldbach pair count $G(2N)$ is consistent with the
Hardy--Littlewood prediction
\begin{equation}\label{eq:HL}
  G_{\mathrm{HL}}(2N)
  \;\sim\; 2C_2
  \prod_{\substack{p \mid N \\ p > 2}}
  \frac{p - 1}{p - 2}
  \;\cdot\;
  \frac{N}{(\ln N)^2}\,,
\end{equation}
where $C_2 = \prod_{p > 2}(1 - 1/(p-1)^2) \approx 0.6601$
is the twin prime constant.  Figure~\ref{fig:HL}(a)
shows the actual--predicted comparison; the scatter around the
diagonal is typical of the known fluctuations in prime pair counts.

\begin{figure}[H]
  \centering
  \includegraphics[width=\textwidth]{../figures/fig_HL_cv.pdf}
  \caption{%
    (a)~Actual Goldbach count $G(2N)$ vs.\
    Hardy--Littlewood prediction.
    (b)~Coefficient of variation $\sigma(\rho)/\langle\rho\rangle$
    vs.\ $2N$, decreasing as the signal-to-noise ratio improves.}
  \label{fig:HL}
\end{figure}


% ═══════════════════════════════════════════════════════════════════════════
\section{Asymptotic Analysis}\label{sec:asymp}

\subsection{Why the BSL works: a quantitative explanation}

The BSL models $\rho$ as a linear function of the static
conduit $\xi = 2\log\rad_{\mathrm{odd}}(N)/\log(2N)$.
From the decomposition~\eqref{eq:decomp}, the ``non-$\xi$''
part of $\rho$ is $\rho_{\mathrm{boundary}} + \delta$.
The BSL achieves $R^2 > 0.997$ because:

\begin{enumerate}
\item
  $\rho_{\mathrm{boundary}} = 2\log(pq)/\log(2N)$ is
  tightly concentrated for fixed~$N$: most Goldbach pairs
  have $pq$ within an order of magnitude of~$N^2$, so
  $\rho_{\mathrm{boundary}}$ varies by $\lesssim 1$.
\item
  The gap correction $\delta$ has bounded standard deviation
  ($\sigma \approx 0.33$), contributing only scatter within
  the band, not systematic drift.
\item
  Together, the systematic part
  ($\rho_{\mathrm{static}} + \langle\rho_{\mathrm{boundary}}\rangle$)
  accounts for $\sim$72.5\% of $\langle\rho\rangle$, and
  this systematic part is a deterministic function of~$N$.
\end{enumerate}


\subsection{The normalised ratio}

Define the normalised conductor ratio
\[
  R(N) \;=\;
  \frac{\langle\rho_{\mathrm{true}}\rangle(N)}{\ln(2N)}.
\]
The census data (Figure~\ref{fig:bw}(b)) shows $R(N)$
decreasing from $\sim\!1.15$ at $2N = 50$ to $\sim\!0.70$
at $2N = 10{,}000$.  We conjecture:

\begin{conjecture}\label{conj:asymp}
  As $N \to \infty$,
  \[
    \langle\rho_{\mathrm{true}}\rangle(N)
    \;=\;
    4 + \frac{2\log\rad_{\mathrm{odd}}(N)}{\log(2N)}
    + O\!\left(\frac{\log\log N}{\log N}\right).
  \]
\end{conjecture}

\noindent
The leading term~$4$ arises because for a ``typical'' Goldbach
pair, $\log(pq) \sim 2\log N$, so
$\rho_{\mathrm{boundary}}$
converges to~$4$.  The second term is exactly the static
conduit~$\xi$.  The gap correction
$\langle\delta\rangle$ should be
$O(\log\log N / \log N)$ on average, since by the
Erd\H{o}s--Kac theorem the average number of distinct prime
factors of a random integer near~$N$ is
${\sim}\,\log\log N$, and each contributes
$O(1/\log N)$ to the normalised ratio.


\subsection{Bandwidth and the Erd\H{o}s--Kac connection}

The observed constancy of $\sigma(\rho) \approx 0.33$ can be
heuristically explained as follows.  The dominant source of
variance in $\rho$ (for fixed~$N$) is the variation in
$\log(pq)$ and $\log\rad_{\mathrm{odd}}^{\mathrm{new}}(|p-q|)$
across Goldbach pairs.  For the boundary term, the variance
comes from the distribution of $\log(pq)$, which for
$p + q = 2N$ is essentially $\log(p(2N - p))$.
This is a concave function with bounded variance regardless
of~$N$.  For the gap term, the Erd\H{o}s--Kac theorem implies
$\omega(|p - q|) \sim \log\log N$ with standard deviation
$\sim \sqrt{\log\log N}$, and each prime factor contributes
$2\log r / \log(2N) = O(1/\log N)$ to~$\delta$.  The net
variance in~$\delta$ is thus $O(\log\log N / (\log N)^2)$,
which is negligible.  The total bandwidth is therefore dominated
by the boundary term and stays bounded.


% ═══════════════════════════════════════════════════════════════════════════
\section{Summary of Census Statistics}\label{sec:table}

\begin{table}[H]
\centering
\begin{tabular}{rrrrrrr}
\toprule
$2N$ & $G(2N)$ & $\langle\rho\rangle$ & $\sigma(\rho)$
& $\langle\rho\rangle/\!\ln(2N)$ & CV \\
\midrule
50     &   4 & 4.517 & 0.130 & 1.155 & 0.029 \\
100    &   6 & 4.992 & 0.349 & 1.084 & 0.070 \\
200    &   8 & 4.845 & 0.448 & 0.914 & 0.092 \\
500    &  13 & 5.144 & 0.305 & 0.828 & 0.059 \\
1000   &  28 & 5.160 & 0.369 & 0.747 & 0.072 \\
2000   &  37 & 5.299 & 0.338 & 0.697 & 0.064 \\
5000   &  76 & 5.349 & 0.301 & 0.628 & 0.056 \\
10000  & 127 & 5.377 & 0.323 & 0.584 & 0.060 \\
\bottomrule
\end{tabular}
\caption{Census statistics at selected values of $2N$.
  $G(2N)$ is the number of Goldbach pairs; CV is the
  coefficient of variation $\sigma/\langle\rho\rangle$.}
\label{tab:census}
\end{table}


% ═══════════════════════════════════════════════════════════════════════════
\section{Discussion}\label{sec:discussion}

The census reveals three structural properties of the conductor
landscape:

\emph{Signal dominance.}
The systematic part of $\rho$ (static conduit $+$ mean boundary)
accounts for $\sim$72.5\% of the total, explaining the BSL's
high~$R^2$.

\emph{Noise boundedness.}
The bandwidth $\sigma(\rho) \approx 0.33$ is bounded, so
the signal-to-noise ratio improves with~$N$.  This is a
necessary (though not sufficient) condition for the conductor
proxy to remain faithful in the asymptotic regime.

\emph{Hardy--Littlewood consistency.}
The Goldbach pair counts follow the Hardy--Littlewood density
with expected fluctuations.  The conductor framework introduces
no anomalies: curves with ``extreme'' conductor ratios do not
cluster at particular~$N$.

These properties do not prove the Goldbach conjecture, but they
establish that the arithmetic geometry of the Frey curves is
smoothly compatible with the conjectured prime pair statistics.
Any approach to Goldbach through conductor methods must ultimately
show that the \emph{existence} of at least one pair forces a
non-trivial conductor, connecting the density picture to an
existence argument.  This remains an open problem.


% ═══════════════════════════════════════════════════════════════════════════
\section*{Acknowledgments}

All computations were performed in Python using a prime sieve
up to 20,002.  Data and scripts are available at\\
\url{https://github.com/Ruqing1963/goldbach-conductor-census}.

\begin{thebibliography}{9}

\bibitem{Chen2026TCV}
R.~Chen,
\emph{The true conductor of Goldbach--Frey curves: computational
  validation of the conductor proxy},
Zenodo, 2026.
\url{https://zenodo.org/records/18749731}

\bibitem{Chen2026UTS}
R.~Chen,
\emph{Universal tame semistability of Goldbach--Frey Jacobians:
  a proof of $f_r = 2$ at all odd primes},
Zenodo, 2026.
\url{https://zenodo.org/records/18751169}

\bibitem{Chen2026DS}
R.~Chen,
\emph{Dynamic stability of the Goldbach locus: conductor orbit
  propagation and the Band Shifting Law in $\GSp(4)$},
Zenodo, 2026.
\url{https://zenodo.org/records/18724884}

\bibitem{HardyLittlewood}
G.\,H.~Hardy and J.\,E.~Littlewood,
Some problems of ``Partitio Numerorum''; III:\ On the expression
of a number as a sum of primes,
\emph{Acta Math.}~\textbf{44} (1923), 1--70.

\bibitem{ErdosKac}
P.~Erd\H{o}s and M.~Kac,
The Gaussian law of errors in the theory of additive number
theoretic functions,
\emph{Amer.\ J.\ Math.}~\textbf{62} (1940), 738--742.

\end{thebibliography}

\end{document}
